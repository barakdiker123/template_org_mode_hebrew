% Created 2024-02-25 Sun 14:16
% Intended LaTeX compiler: lualatex
\documentclass[11pt]{article}
\usepackage{amsmath}
\usepackage{fontspec}
\usepackage{graphicx}
\usepackage{longtable}
\usepackage{wrapfig}
\usepackage{rotating}
\usepackage[normalem]{ulem}
\usepackage{capt-of}
\usepackage{hyperref}
\usepackage[bidi=basic, english]{babel}
\babelprovide[main, import]{hebrew}
\babelfont{rm}{Libertinus Serif}
\usepackage[margin=1in]{geometry}		% For setting margins
\usepackage{amsmath}				% For Math
\usepackage{fancyhdr}				% For fancy header/footer
\usepackage{graphicx}				% For including figure/image
\usepackage{cancel}					% To use the slash to cancel out stuff in work
\pagestyle{fancy}
\fancyhead[LO,L]{ברק דיקר }
\fancyhead[CO,C]{ECE305 - Homework Example}
\fancyhead[RO,R]{\today}
\fancyfoot[LO,L]{}
\fancyfoot[CO,C]{\thepage}
\fancyfoot[RO,R]{}
\renewcommand{\headrulewidth}{0.4pt}
\renewcommand{\footrulewidth}{0.4pt}
\setcounter{secnumdepth}{0}
\author{Barak-Nadav Diker}
\date{\today}
\title{אלגברה ליניארית}
\hypersetup{
 pdfauthor={Barak-Nadav Diker},
 pdftitle={אלגברה ליניארית},
 pdfkeywords={},
 pdfsubject={},
 pdfcreator={Emacs 29.2 (Org mode 9.7)}, 
 pdflang={English}}
\begin{document}

\maketitle
\newpage
\section*{שאלה 1}
\label{sec:org9393a63}
\subsection*{סעיף א}
\label{sec:orgb6a40ad}
יהי
\(A,B \in \mathbb{M} _{n*n}\)
מטריצות משולשות עליונות

הראו כי המטריצה
\(A+B\)
היא מטריצה משולשת
בעזרת שימוש באינדקסים
\subsection*{סעיף ב}
\label{sec:org7fe3ca9}
יהי
\(A,B \in \mathbb{M} _{n}\)
מטריצות משולשות עליונות

הראו כי המטריצה
\(A*B\)
היא מטריצה משולשת
בעזרת שימוש באינדקסים
\section*{שאלה 2}
\label{sec:org077ed1d}
יהי
\(A\in M_{2} (\mathbb{R})\)
אם
\(A^2 = A\)
אז הערכים היחידים האפשריים עבור
\(tr(A)\)
הם
1,2,0
תנו דוגמאות למטריצות בהם הערכים האלה אכן מתקבלים
\section*{שאלה 3}
\label{sec:org67cb190}
יהיו

\(A,B,C \in M_n(\mathbb{R})\)

מטריצות ריבועיות מסדר
\(n\)
על
\(n\)
ענו על הסעיפים הבאים
\subsection*{סעיף א}
\label{sec:orgf77c292}
הראו כי

\(tr(ABC) = tr(BCA) = tr(CAB)\)
\subsection*{סעיף ב}
\label{sec:orge6a2c80}
תנו דוגמא שבה
\(tr(ABC) \neq tr(CBA)\)
מתקיים
\subsection*{סעיף ג}
\label{sec:org73e12a3}
הוכיחו שאם
\[A,B,C\]
מטריצות סימטריות אז גם
מתקיים
\[ tr(ABC) = tr(CBA) \]
\section*{שאלה 4}
\label{sec:orga49d475}
יהיו
\(A,B\)
מטריצות סימטריות
\subsection*{סעיף א}
\label{sec:org495b52f}
הראו כי
\(AB\)
סימטרית
אם ורק אם
\(AB-BA = 0\)
מתקיים
\subsection*{סעיף ב}
\label{sec:org3dd6936}
הראו כי
\[ AB+BA \]
תמיד סימטרית
\section*{שאלה 5}
\label{sec:org8d4fd92}
יהיו
\(A,B\)
מטריצות ריבועיות אנטי-סימטרית
\subsection*{סעיף א}
\label{sec:org84ba353}
הראו כי
\(AB\)
אנטי סימטרית
אם ורק אם
\(AB+BA = 0\)
מתקיים
\subsection*{סעיף ב}
\label{sec:orgbad8eda}
הראו כי
\(AB-BA\)
תמיד אנטי סימטרית
\section*{שאלה 6}
\label{sec:org4d35995}
יהי
\(A\)
מטריצה סימטרית
\(B\)
מטריצה אנטי-סמטרית

הראו כי
\[tr(AB) = 0 \]
\end{document}
