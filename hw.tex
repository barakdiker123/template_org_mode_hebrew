% Created 2024-02-23 Fri 16:14
% Intended LaTeX compiler: lualatex
\documentclass[11pt]{article}
\usepackage{amsmath}
\usepackage{fontspec}
\usepackage{graphicx}
\usepackage{longtable}
\usepackage{wrapfig}
\usepackage{rotating}
\usepackage[normalem]{ulem}
\usepackage{capt-of}
\usepackage{hyperref}
\usepackage[bidi=basic, english]{babel}
\babelprovide[main, import]{hebrew}
\babelfont{rm}{Libertinus Serif}
\setcounter{secnumdepth}{0}
\author{Barak-Nadav Diker}
\date{\today}
\title{אלגברה ליניארית}
\hypersetup{
 pdfauthor={Barak-Nadav Diker},
 pdftitle={אלגברה ליניארית},
 pdfkeywords={},
 pdfsubject={},
 pdfcreator={Emacs 29.2 (Org mode 9.7)}, 
 pdflang={English}}
\begin{document}

\maketitle
\tableofcontents

\section*{שאלה 1}
\label{sec:org6f1ba3b}
\subsection*{סעיף א}
\label{sec:org587440c}
יהי
\(A,B \in \mathbb{M} _{n*n}\)
מטריצות משולשות עליונות

הראו כי המטריצה
\(A+B\)
היא מטריצה משולשת
בעזרת שימוש באינדקסים
\subsection*{סעיף ב}
\label{sec:org0b6de0c}
יהי
\(A,B \in \mathbb{M} _{n*n}\)
מטריצות משולשות עליונות

הראו כי המטריצה
\(A*B\)
היא מטריצה משולשת
בעזרת שימוש באינדקסים
\section*{שאלה 2}
\label{sec:orga5a3662}
יהי
\(A\in M_{2} (\mathbb{R})\)
אם
\(A^2 = A\)
אז הערכים היחידים האפשריים עבור
\(tr(A)\)
הם
1,2,0
תנו דוגמאות למטריצות בהם הערכים האלה אכן מתקבלים
\section*{שאלה 3}
\label{sec:orge1eafbf}
יהיו

\(A,B,C \in M_n(\mathbb{R})\)

מטריצות ריבועיות מסדר
\(n\)
על
\(n\)
ענו על הסעיפים הבאים
\subsection*{סעיף א}
\label{sec:org661b40e}
הראו כי

\(tr(ABC) = tr(BCA) = tr(CAB)\)
\subsection*{סעיף ב}
\label{sec:org1f4f788}
תנו דוגמא שבה
\(tr(ABC) \neq tr(CBA)\)
מתקיים
\subsection*{סעיף ג}
\label{sec:orgcd28f55}
הוכיחו שאם
\[A,B,C\]
מטריצות סימטריות אז גם
מתקיים
\[ tr(ABC) = tr(CBA) \]
\section*{שאלה 4}
\label{sec:org6cbfff0}
יהיו
\(A,B\)
מטריצות סימטריות
\subsection*{סעיף א}
\label{sec:org963335f}
הראו כי
\(AB\)
סימטרית
אם ורק אם
\(AB-BA = 0\)
מתקיים
\subsection*{סעיף ב}
\label{sec:org055d8e1}
הראו כי
\[ AB+BA \]
תמיד סימטרית
\section*{שאלה 5}
\label{sec:orgb2b981b}
יהיו
\(A,B\)
מטריצות ריבועיות אנטי-סימטרית
\subsection*{סעיף א}
\label{sec:orga24f5d1}
הראו כי
\(AB\)
אנטי סימטרית
אם ורק אם
\(AB+BA = 0\)
מתקיים
\subsection*{סעיף ב}
\label{sec:org345382f}
הראו כי
\(AB-BA\)
תמיד אנטי סימטרית
\section*{שאלה 6}
\label{sec:org167bc99}
יהי
\(A\)
מטריצה סימטרית
\(B\)
מטריצה אנטי-סמטרית

הראו כי
\[tr(AB) = 0 \]
\end{document}
